\documentclass[a4paper,11pt]{article}
\usepackage[T1]{fontenc}
\usepackage[utf8]{inputenc}
\usepackage{lmodern}
\usepackage{hyperref}
\title{COMP2}
\author{}

\begin{document}

\maketitle
\tableofcontents
\newpage

\section*{Review Partiel}

Difficultes sur regle d'inference. (Revoir les regles d'inference sur les types)\\\\
ebreak(\texttt{f()}) , si \texttt{f() = readint()} alors c'est impossible de binder  a la compilation.\\\\
Coercission  aggrandissante, on stocke la variable dans un champs contenant plus de bits.\\
Coercission retrecissante, on stocke la variable dans un champs contenant moins de bit, Avec de la perte d'info.

\section{Langage intermediaire}

\subsection{Pourquoi langage intermediaire ?}
Reduire la quantite de code a ecrire, et les duplications de code.\\\\
Il faut trouver un point de convergence entre differents langage afin de produire  un langage intermediaire.\\
Soit on le rapproche de la partie assembleur, soit on le rapproche des langages de plus haut niveau.\\

On a 4 etapes pour faire un compilateur :\\
- FrontEnd\\
- HIR (High Level Representation)\\
- LIR (Low Level Representation)\\
- Backend\\

Si on veut faire des optimisations, il faut les faire entre HIR et LIR, ainsi, cela profitera a tous les langages et tous les backends.\\

\subsection{Comment savoir quand on est dans la representation intermediaire ?}
A  partir du moment ou le langage intermediaire est defini, on peut le savoir.\\
Il faut s'assurer que notre traduction est correcte. Dans ce cas la le pretty printer est utile.
Il y  a differentes manieres de faire la representation intermediaire:\\
- fonctionelle\\
- imperative\\

Cela depend des langages a traduire.

\subsection{GCC}\hfill \break
GCC traduit differents langages vers GENERIC qui est un premier langage intermediaire.\\
Apres il simplifie avec gsimplifier.\\
Apres il ne va avoir que des variables constantes. Donc beaucoup plus de variable mais avec un temps de vie plus petit.\\

\subsection{Stack ou Registre}
Le langage intermediaire va s'executer en utilisant soit des registres, soit la stack.

\subsection{Langage impur/ langage pur}
Dans un langage impur, on va avoir des effets de bord. Donc le retour de la fonction appelee est cast en void.
Dans un langage pur, cette fonction ne peut pas exister.

\subsection{Pourquoi s'executer dans un registre ?}

Pour avoir des performances optimales.\\
Cependant le nombre de registres est limite.\\
Dans l'ideal il faudrait utiliser uniquement les registres.\\
En cas de manque de registre on peut utiliser les caches L1 et L2.\\
Ou la RAM.\\
Ou le disque.\\

Cependant plus l'on descend dans cette liste , plus l'execution va etre lente.\\

Pour les registres, l'ordre des arguments que l'on va passer dans les registres ne doit pas etre neglige. C'est le fondateur du langage qui decide de cette organisation.\\

\subsection{Actions sur la Pile}
Rappel du cours d'assembleur:\\
On va sauvegarder fp (frame ptr) et sp (stack ptr) dans la stack pour pouvoir les restaurer a la fin des appels de fonction.\\
Regle a respecter : je n'ai pas le droit d'acceder a de la memoire en dehors de ma frame, je peux cependant l'aggrandir.\\

\section{Frame n recursion}

\subsection{recursion}
On doit garder un static link sur les frames precedentes.

\

\
\begin{tabular}{|l|}
\hline
ONE   \\ \hline
TWO   \\ \hline
THREE \\ \hline
\end{tabular}
\\

\
TWO garde un static ptr sur ONE\\et THREE garde un static ptr sur TWO.

\

\
\begin{tabular}{|l|}
\hline
ONE   \\ \hline
TWO   \\ \hline
TWO   \\ \hline
TWO   \\ \hline
TWO   \\ \hline
TWO   \\ \hline
THREE \\ \hline
\end{tabular}
\\

\
chaque TWO garde un static ptr sur ONE\\et THREE garde un static ptr sur le TWO le plus profond.

\subsection{MIPS}
En MIPS le static link est stocke au debut de la frame.
\subsection{Assembleur}
Pour la partie Hir Level, on va devoir sauvegarder les variables locales par frame. \\
La connaissance de la taille de la frame sera faite a l'execution.\\


\subsection{Linearisation}

On va chercher a applatir l'arbre construit par le compilateur.\\

\newpage
\subsection{CJUMP}
\textemdash \textemdash \textemdash\textemdash \textemdash \textemdash \\
\texttt{if condition\\ then\\ do A\\ else\\ do B}\\
\textemdash \textemdash \textemdash\textemdash \textemdash \textemdash \\
$\Longleftrightarrow $\\
\textemdash \textemdash \textemdash\textemdash \textemdash \textemdash \\
\texttt{JumpEqual condition labelA\\
labelB (Optional)\\
... (do B)\\
Jump labelendAB\\
labelA\\
...(do A)\\
labelendAB\\
}
\textemdash \textemdash \textemdash \textemdash \textemdash \textemdash \\







\end{document}
