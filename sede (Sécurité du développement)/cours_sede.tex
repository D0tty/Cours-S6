\documentclass[a4paper,11pt]{article}


% Standart.
\usepackage[T1]{fontenc}
\usepackage[utf8]{inputenc}
\usepackage{lmodern}
\usepackage[english,french]{babel}

% Defining sane commands for euros.
\usepackage{textcomp}
\newcommand{\euro}{\texteuro}

% Page format.
\usepackage{geometry}
\geometry{top=2cm, bottom=2cm, left=1cm, right=1cm}

% Figures.
\usepackage{graphicx}
\graphicspath{{resources/}}
\usepackage{float}
\usepackage{array}
\usepackage{tikz}

% Source code
\usepackage{minted}

% Links. Should be loaded last.
\usepackage{hyperref}
\hypersetup{colorlinks=true,
            linkcolor=blue,
            filecolor=magenta,
            urlcolor=cyan,
            citecolor=blue}



\title{Software Development Security}
\author{Marc ESPIE}
\date{14 mai 2018, promo 2020}

\begin{document}
\selectlanguage{english}

\maketitle

\tableofcontents
\newpage

\section{BASIC Security}

One can get the slides at \url{https://www.lse.epita.fr/teaching/courses.html}

\subsection{Introduction}
\subsubsection{Perpectives}
There are more conferences for attackers than for safety.\\

Perspectives :\\
- basic ecosystem of software from security perspective\\
- vocabulary to pass intership\\
- Dispell misconceptions about dev security\\

Setting limits :\\
- C and Unix\\
- All the bases of programs are in C or assembly languages\\

Unix is easier to start with, it's well known and well defined
it has existed for more than 20 years. Then it will be not that hard to go on other OSs.\\

BOOKS:\\
- Building Secure Software(Viega, McGraw, ISBN  0-201-72152-X)\\
- Open BSD papers\\

\subsubsection{Coming Exam}

Simple questions for the exam:\\
- You will have access to lecture notes\\
- So if I ask you to define a term, you shouldn't copy your notes\\
- You should be able demonstrate that yoou understand the term by explaining it your own words
- Give concrete examples\\
- Create your own examples\\

Adanced Questions:\\
- There will be source code and sample audits\\
- It won't be 100\% clean\\
- It won't be exactly like 'epita standard code'\\
- If it's different it may not be wrong\\
- Beware of wrong asssumptions\\
- The security issues to fix will be nasty ones\\

Exam in general:\\
- you can write wether in french  or english\\

\subsection{Classical companies}
\subsubsection{Classical way companies write specs}
One person by task\\

But it's not a good idea.\\

\subsubsection{Auditors}

If an auditor finds a bug\\
...Sometime it's because the design is wrong\\
Auditors can't catch all\\
... So devs must know about good practices\\

You can't have only one developer to remove all the security breaches
it's too much.\\

\subsubsection{Testers}

You can't have pit testers VS devellopers.
A good tester is invaluable.


\subsubsection{Sadly}
A lot of good databases experts don't even know about SQL injections.



\subsection{For a release}

\subsubsection{Mozilla }
When Mozilla ship a new version of firefox, they give the code.\\
But they have to adapt it for each OS.\\
So they do modify it before.\\

So after the realease, they are branchings.\\
Some people have to still work on correcting bugs.\\

When working on bugs, you can produce more bugs.\\

At some point you have to say that some versions are no more supported.\\

\subsubsection{Life of a product}
EOL End of life for a product\\
ESR Extended support release\\

For Windows XP, microsoft had to support for a very long time this version, because companies needed to use it and didn't wanted to change it.\\

Companies prefer to have a stable version of a program with security breach a program less stable but with security corrected.\\


\subsection{Bugs}
\subsubsection{Security and bugs}

A bug is not a 'security Hole'.\\
Most attacks are a serie of bugs.\\
We want to have defense in depth.\\
Fixing one bug stops the attack!\\
An attack is also called an exploit.\\
Software has vulnerabilities.\\



\subsubsection{Finding a bug}

- A developer can find a bug.\\
- External user find the bug and report it to the company.\\
- Others can sell this bug.\\


\subsubsection{What to do}

Be proactive about a security issue.\\
Fixing it without letting the bad guys know.\\

A soon as you find something, you have to fix it, because there will be an soon or later.\\


\subsubsection{Little history}

25 years ago, there was a mailing list called bugtraq, but multiple times you had the same new on different OSs.\\
Now there is CVE\\
CVE : common vulnerabilities and exposures.\\

\subsubsection{ Reporting a bug}
Don't do it on Friday\\
Account for vendors\\
Have a "secure" channel for bugs\\


"Meltdown" for Intel processors : is a good example of not what to do.\\


\subsubsection{Worst case scenario : zero days}
It's when you see people exploiting a bug.



\subsubsection{ Bad behaviors}
- It's too complicated to be exploitable\\
It will take time but it will be exploited.\\

The IE5 url overflow :\\
Use it to craft assembly code to craft more overflow.\\
Two month later, a guy used it to write self modifying code.\\


\subsubsection{ Be prepared}

Sofware components get reused all the time.\\
Plan to be successfull.\\

Your code may be used in Hospitals, so nasty guys could exploit it and could kill people.


\subsubsection{ Closed source}
It's no more secure.\\
lot's of people know how to reverse engineer.\\

You want to avoid the 'sweep under the carpet effect'\\

If you do open source, some people will look at what you are doing.\\

example:\\
\
A guy recovered most patterns of buffer overflow from microsoft windows.\\
\
It takes one bug...\\
Everything is exploitable.\\
\
You have to carefully check the code that you have been writting.\\


\subsubsection{ How to avoid bugs}
- Don't do bugs\\
- Know you APIs\\
- Prefer secure idioms\\
\subsection{Avoiding bug exploit}
\subsubsection{Mitigation}
In OS, it's an actual technique that will mitigate it.\\
\
example : canaries\\
Function prolog will insert random data on the stack.\\
This will be checked at the end of the function, if it has changed then, stack smash if called to make program end.\\


\subsubsection{ Guard pages}

Pages are separed in order to avoid writting on other pages.


\subsubsection{ Better API's}

Don't use strcpy, strcat, strncpy nor strncat.\\
prefer strlcpy, strlcat; so you can use the size to see if it will fit.\\


\subsubsection{ The Drepper fallacy}
- 'But I don't write wrong code'\\
- The reason for slow adoption of strlcpy\\
For over 10 year Mr Drepper didn't wanted to introduce strlcpy on linux, because people had to write good stuff.\\
\
Prefer snprint to sprintf.\\
When writing code in C, the size have to be obvious.\\
\
90\% of all software is:\\
- crap\\
- unimportant to optimize\\
- etc...\\
\
You can't fix everything\\
... therefore don't fix everything\\
This produces "Low-Hanging fruits"\\
As long as you avoid basic bugs, you're safe from most pirates. (95\% of them)\\

\section{Unsecure programs}
\subsection{Buffer Overflow}
Be carefull with malloc overflows.\\
The programm opening an image can be vulnerable to overflows.\\
\
ie: flash with buffer overflows. If you did the right overflow, you could have get more rights than you should have had.\\
\
You have to write correctly your library so it is fast and safe.\\
\
Beware of undefined behaviors and compiler optimizations.\\
\
When working with arrays in C, you want to check the size.\\
Interger overflow => undefined behavior\\
\
You have to check th size of max integer depending on your CPU.\\

Broken code:

\begin{minted}{c}
  int *
  alloc_array(int n)
  {
    int *t = emalloc(n * sizeof(int));
    return t;
  } 
  int * 
  read_array() 
  { 
    int s = 0; 
    scanf("%d", &s); 
    if (s == 0) 
    exit(1); 
    int *t = alloc_array(s); 
    for (int i = 0; i != s; i++) 
    scanf("%d", t[i]); 
    return t; 
  }
\end{minted}

Good practices:

\begin{itemize}
\item Use calloc. Make sure that it is safe.
\item Maybe craft you own.
\end{itemize}

\subsection{SQL injection}
In SQL, never use do.

Don't assume the people you work with know about this.\\
Never do matching about negative patterns.\\
ie : Some patterns can be added later, and so you won't be protected anymore.\\

You should accept the less things you can, and open more things when people ask you to do it. In order to avoid security leaks.

\subsection{More about quoting}

What you don't know will kill you

Never do matching against negative patterns

e.g., an email address is not something that does not contain some characters

Is is something that only matches a given pattern.

(Subsidiary question: figure out a regexp that matches email)

\

Do positive matching.

\subsection{Printf}
What you don't know will kill you.\\
Printf can print part of memories that you don't want to share.\\
Beware of executing shell commands from user or asking for paths.\\

Always write printf("\%s", msg);

The compiler will optimize it.

\subsection{Forgotten shells}

Every time one launches system or popen, a shell is run.

\subsection{Services}
When a service crash, what do you do ?\\
-restart service\\
-don't restart it\\

The good idea is not to restart it yet, and look at the logs.\\
So you can know where is the problem.\\
There may be attackers.\\

If you  have a system that gives you notifications or warning messages, you have to use them.\\
\
You should try to attack your owns servers, and make them fail, in order to be sure they fail for the good reasons.\\
\
Make sure that you verify regularly that things can fail.\\
\
Netflix had some of the same issues, it was client of AWS.\\
\url{https://www.lemondeinformatique.fr/actualites/lire-netflix-libere-chaos-monkey-dans-la-jungle-open-source-49940.html}

\subsection{More sociology}

Usually, the finder of the exploit will let in one or two syntax errors. Script
kiddies catch them and fix them and abuse the thing.

They usually do a rm -rf * at the root of the fs...

Keep the logs outside (on a read-only machine) and make backups frequently.

After all, consider how expensive it is to do them and test themand how
expensive it is to restart the process.

\section{Trusting code}

\subsection{Open source}
If you want people to find bugs in your code, you have to write well documented.

Sometime, people don't look enough. ``Many eyeballs fallacy''. They found a
20-year-old bug in the OpenBSD libc.

You still have to be really carefull about your stuff.

Code reviews are a good thing. Also make the code readable. Being read by pairs
is really gainful.

Trusting people can be dangerous. Being paranoic is a good thing.

One man added a security hole in ssl in python, he was redirecting data to his
website. In ssh-decorate. Israeli developer. He detended it wasn't his fault and
was hacked. He could also be kidding with ourselves. He removed his software.

How to protect ourselves? Checksums.

Be careful when you do things like wget http://truc.com/bidule.html | bash...

Running code inside a docker is not enough. It can probably access network
things, make some shitty processing...

\subsection{Obstacles}

The systems that give you ``just-in-time'' tarballs. Make releases and huge
version numbers.

... host them elsewhere.

\subsection{Generated files}

Typically autoconf and automake.

They can be badly generated, and be source of security holes.

If you are using tools that generate code, you have to give the steps to
regenerate files.

Never change by hand generated files, it's a very bad practice.

There have already been trojans in the configure files.

It makes it hard to have reliable builds.

We seldom can re-generate the same configure.

Autoconf will compile all it can by default. Do compilation diffs.

\subsection{Guidelines}

Always make it possible to regenerate everything

... so that people may audit stuff

build should not have access network access.

... and probably have network access.

For instance, in OpenBSD, we switched to doing that, and we caught python/ruby
code accessing the network.


... no recent autoconf/automake trojan.

\subsection{Race condition example}

mktemp example: it does not create the file, only returns a valid file NAME.

Every once in a while, you can have conflicts.

You can overwrite config files, etc.

Use counting.

\subsection{Race condition}

Trying to access a shared resouce using non-atomic operations.

/tmp is a common directory

mktemp checks the file does not exist

fopen assumes the file still does not exist.

Use mkstemp instead.

\subsection{Correct use of mkstemp}

* Remove the fd when your code fails.

* errno can change. Save it.

\subsection{Yet another example}

Config file opener:

Use atomic operations. fstat after fopen.

\subsection{Solutions}

Know atomic operations.

Prefer fstat, fchmod, fchown to stat, chmod, chown...

ldconfig example: remove the first part. rename will delete the old files.

\subsection{ERRNO}

You have to be carefull about erno usage. If you do a syscall between a failed
syscall and err(), you will report the syscall you did in the middle.

\subsection{Yet another}

No errno restauration in a signal handler.

printf works with a buffer. Buffer overflow possible.

Careful with memory allocation in signal handlers.

Never use wait (waitpid is ok).

\end{document}
