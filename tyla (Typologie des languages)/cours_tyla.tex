\documentclass[a4paper,11pt]{article}

% Standart.
\usepackage[T1]{fontenc}
\usepackage[utf8]{inputenc}
\usepackage{lmodern}
\usepackage[french]{babel}
\usepackage{eurosym}

% Page format.
\usepackage{geometry}
\geometry{top=2cm, bottom=2cm, left=1cm, right=1cm}

% Figures.
\usepackage{graphicx}
\graphicspath{{resources/}}
\usepackage{float}
\usepackage{array}
\usepackage{tikz}

% Math and chemistry.
\usepackage{chemist}
\usepackage{siunitx}
\usepackage{numprint}
\usepackage{amsmath}
\usepackage{amssymb}
\usepackage{stmaryrd}

% Links. Should be loaded last.
\usepackage{hyperref}
\hypersetup{colorlinks=true,
            linkcolor=blue,
            filecolor=magenta,
            urlcolor=cyan,
            citecolor=blue}


% Useful environment
\newenvironment*{dummyenvironment}{}{}

% Custom commands
\newcommand{\R}{\mathbb{R}}
\newcommand{\Z}{\mathbb{Z}}
\newcommand{\N}{\mathbb{N}}
\newcommand{\der}{\,\mathrm{d}}
\newcommand{\e}{\mathrm{e}}
\newcommand{\ti}{\cdot}

\title{Typologie des languages}
\author{Étienne RENAULT}
\date{20 mars 2018}

\begin{document}

\maketitle
\tableofcontents
\newpage
\section{Historique}

\subsection{Chronologie}

Vérifier les dates!

\begin{itemize}

\item Première machine pour calculer: abaque. (addition, soustraction)

\item 1643: Pascaline, première machine à calculer mécanique. (addition,
  soustraction)

\item 1673: Première machine avec multiplications et divisions.

\item 1725: Basille Bouchon, mécanisation d'un métier à tisser. Transposition
  d'un programme sur un support. La machine n'a pas duré faute de fragilité.

\item 1780: Découverte de l'électricité par Benjamin Franklin. (Avec cerf-
  volant).

\item 1805: Jacquard fait un métier à tisser complètement automatisé à carte
  perforée   et alimenté à l'électricité.

\item 1833: Babbage: machine analytique. Sortie non-tissu. Sortie sur des roues
  crantées. Ada Lovelace, comtesse, première programmeuse. Elle faisait de la
  science, rare à l'époque. Elle fait de la traduction de sciences pour la
  machine de Babbage.

\item 1876: Téléphone, par Alexander Graham Bell et Elisha Gray (stagiaire).
  Le stagiare fait tomber un truc  (?)

\item 1911: CTRC (ancêtre d'IBM). Doit toquer chez tous les Américains pour
  faire un resencement.
  IBM a construit des machines pour faire le recensement.

\item 1924: IBM prend son nom.

\item 1927: Première démonstration publique d'une télévision.

\item 1936: Premier calculator, le Z1. Konrag Suze. Fait un ordinateur chez
  ses parents. machine détruite par les Allemands pendant la guerre.
  Reconstruite après par le même.

\item 1939: Radio Shack (catalogue).

\item 1939: Conception de l'ABC (Atanasoff-Berry Computer). Ruled the first automatic
  digital computer en 1973. Not programmable, not turing-complete.

\item 1940: Première TV couleur.

\item 1941: Première machine Turing-complète: Zuse's Z3. Basée sur des tubes à
  vide.

\item 1944: March 1. Machine pour de la défense militaire. 12 000 tubes à vide.
  Une chance sur 2 pour qu'un tube à vide avant la fin de l'exécution du
  programme.

\item 1945: Grace Hopper: trouve le premier bug sur le Harvard Mark II.
  Invente aussi le COBOL plus tard.

\item 1946: ENIAC, 18 000 tubes.

\item 1946; Univac

\item 1948: Transistor. (Hyper artisanal).

\item 1949: EDVAC: Première machine pour tester des disques magnétiques.

\item 1949: Binac: Premier ordinateur pour diriger des missiles \textbf{en temps
    réel}.

\item 1949: Le MIT construit la première machine qui joue aux échecs.

\item 1949: Premier ordinateur commercialisé (demi-pièce): Univac II.

\item 1952: RCA développe le Bizmac. Première base de données.

\item 1952: Imprimante à matrice pour l'Univac 1107 dans les années 60.

\item 1953:  Bande magnétiques.

\item 1953: IBM 701. Machine à louer. Addition en 5 cycles, comme maintenant.
  Fortran est créé sur ces machines par John Backus. Révolution. On passe de
  zéros et de uns à une feuille de papier avec variables et commentaires.

\item 1954: Premier OS pour l'IBM 704, un mainframe.

\item 1954: RAND corporation: modèle pour illustrer l'ordinateur en 2004.

\item 1955: Premier groupe d'utilisateur: SHARE (IBM 701).

\item 1955: IBM demande à chaque pays de traduire computer. Jacques Perret: ``ordinateur''.

\item 1957: Film ``Desk Set''. Ressemble aux assistants vocaux actuels.

\item 1959: 8 traîtres. Bossent sur tubes à vide nouvelle génération. Partent
  tous en même temps.

\item 1959: Texas: Premier circuit intégré.

\item 1959: COBOL

\item 1959: IBM 401: 10 000 machines vendues.

\item 1960: Téléphone automatisé.

\item 1961: Premier micro-ordinateur. Fourni avec écran.

\item PDP-1: Space war. Premier jeu vidéo.

\item 1963: Tandy achète Radio Shack. Lancement pour le grand public.

\item 1964: Un étudiant du MIT joue de la musique sur le PDP-1.

\item 1964: CDC 6000. Ordinateur le plus puissant de son époque.

\item 1964: Création du BASIC. Language interprété. Interpréteur écrit en
  Assembleur.

\item CDC: Lance des études d'informatique.

\item 1965: Douglas Engelbart: Créé la première souris. Génie. Première
  démonstation enregistrée.

\item Herb Sutter (gourou du C++). Résume toutes les contributions d'Engelbart.
  A aussi inventé le pattern matching.

\item 1965: IBM 360. Couvrait tous les besoins de l'informaticien.

\item 1965: Premier doctorant en informatique.

\item 1968: Première calculette par Texas et HP.

\item 1968: Fondation d'Intel.

\item 1969: Fondation d'AMD.

\item 1969: IBM intrduit System-3.

\item 1969: Bell Labs arrête MULTICS. Ken Thompson implémente alors UNICS sur un
  PDP/7. (4K mots de 18 bits). Codé en P. Jeu de mots avec Eunuque car sa femme
  était partie.

\item 1970: IBM vend

\item 1971: IBM lance les disquettes.

\item 1971: Intel 4004.

\item 1971: Livres grand public.

\item 1971: Kenbak I. 256 bits de RAM. Pour 750\$. Se branche sur la TV.

\item 1972: Atari lance Pong.

\item 1975: MITS lance l'Altair.

\item 1975: Ouvertue de la première boutique d'ordinateurs.

\item 1975: IBM 5100. Flop.

\item 1976: Z80.

\item 1976: Apple I.

\item 1977: Atari VCS 2600.

\item 1977: Bill Gates arrêté pour avoir grillé un feu rouge.

\item 1977: Apple II.

\item 1977: Apple, Comodore et Tnady commencent à vendre des ordianteurs
  personnels.

\item 1978: Premier demi-million d'ordinateurs vendus.

\item Ouverture de The Source et de Compulab.

\item 1980: Sinclair ZX 80 pour 99\$.

\item 1981: IBM PC 5150. Appel d'offre pour les OS (à compléter)

\end{itemize}

\end{document}
